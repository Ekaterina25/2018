\documentclass[12pt]{article} 

\usepackage{ucs} 
\usepackage[utf8x]{inputenc}
\usepackage[russian]{babel} 
\usepackage[left=2cm,right=2cm,top=0,5cm,bottom=0,5cm]{geometry}
\date{}
\title{\bf 17 уравнений, которые изменили мир \\Иэна Стюарта\\} 

\begin{document} 

\maketitle
\begin{enumerate}
 \item Теорема Пифагора \hfill \begin{minipage}[t]{100mm} $a^2+b^2 = c^2$ \hfill Пифагор 530 BC \end{minipage}
 \item Логарифмы  \hfill \begin{minipage}[t]{100mm} $\log xy = \log x+\log y$ \hfill Напиер, 1610 \end{minipage} 
 \item Дифференцильное исчисление \hfill \begin{minipage}[t]{100mm} $\frac{df}{dt} = \lim_{h \to 0}\frac {f(t+h)-f(t)}{h} $ \hfill Ньютон, 1668 \end{minipage}
 \item   Закон гравитации \hfill \begin{minipage}[t]{100mm} $F = G \frac{m_1m_2}{r^2}$ \hfill Ньютон, 1687 \end{minipage}
 \item \noindent
  \begin{minipage}[t]{45mm} Квадратный корень из минус единицы \end{minipage}
  \hfill
  \begin{minipage}[t]{100mm}  $i^2 = -1$ \hfill Эйлер, 1750 \end{minipage}
 \item \noindent
  \begin{minipage}[t]{45mm} Формула Эйлера для многогранников \end{minipage}
  \hfill 
  \begin{minipage}[t]{100mm}  $V - E + F = 2$ \hfill Эйлер, 1751 \end{minipage}
 \item Нормальное распределение \hfill \begin{minipage}[t]{100mm} $\Phi(x) = \frac{1}{\sqrt{2\pi\rho}}e^\frac{(x-\rho)^2}{2\rho^2}$ \hfill Гаусс, 1810 \end{minipage}
 \item Волновое уравнение \hfill \begin{minipage}[t]{100mm} $\frac{\partial^2 u}{\partial t^2} = c^2 \frac{\partial^2 u}{\partial x^2}$ \hfill Д'Аламбер, 1746 \end{minipage}
 \item Преобразование Фурье \hfill \begin{minipage}[t]{100mm} $f(\omega) = \int\limits_\infty ^\infty f(x)e^{-2\pi ix \omega}dx$  \hfill Фурье, 1822 \end{minipage}
 \item 
  \begin{minipage}[t]{45mm} Уравнение \\Навье-Стокса\\
  \end{minipage}
  \hfill
  \begin{minipage}[t]{100mm}$\rho(\frac{\partial {\bf v}}{\partial t}+{\bf v}\cdot\nabla{\bf v} ) = -\nabla p+\nabla\cdot{\bf T}+{\bf f}$ \hfill Навье,Стокс, 1845
  \end{minipage}
   \item Уравнения Максвелла \hfill \begin{minipage}[t]{100mm}  \begin{minipage}[t]{30mm}  $\nabla\cdot{\bf E} =   \frac{\rho}{\varepsilon_0}$ \\ $\nabla\times{\bf E} = -\frac{1}{c} \frac{\partial{\bf H}}{\partial t}$ \\ \end{minipage}
  \begin{minipage}[t]{30mm} $\nabla\cdot{\bf H} = 0$ \\$\nabla\times{\bf H} = \frac{1}{c} \frac {E}{\partial t}$ \\
   \end{minipage}  \hfill Максвелл, 1865\end{minipage}
   \item \begin{minipage}[t]{45mm} Второй закон \\термодинамики\\ \end{minipage}
  \hfill
 \begin{minipage}[t]{100mm}  $dS \ge 0$ \hfill Больцман, 1874 \end{minipage}
  \item Относительность \hfill \begin{minipage}[t]{100mm} $E = mc^2$ \hfill Эйнштейн, 1905
\end{minipage}
\item\begin{minipage}[t]{45mm} Уравнение \\Шредингера\\ \end{minipage}  \hfill \begin{minipage}[t]{100mm} $ih\frac{\partial}{\partial t}\Psi = H\Psi $
    \hfill Шредингер, 1927 \end{minipage}
 \item Теория информации \hfill
  \begin{minipage}[t]{100mm} $H = - \sum p(x)\log p(x)$ \hfill Шеннон, 1949
   \end{minipage}
 \item Теория хаоса  \hfill \begin{minipage}[t]{100mm} $x_t+1 = kx_t(1-x_t)$  \hfill Р.Мэй, 1975 \end{minipage}
   \item \begin{minipage}[t]{45mm} Уравнение \\Блэка — Шоулза\\ \end{minipage}
  \hfill 
 \begin{minipage}[t]{100mm}  $\frac{1}{2}\sigma^2 S^2 \frac{\partial^2 V}{\partial S^2}+rS \frac{\partial V}{\partial S}+\frac{\partial V}{\partial t}-rV = 0$ \hfill Блэк,Шоулз, 1990 \end{minipage}
\end{enumerate}
\end{document}